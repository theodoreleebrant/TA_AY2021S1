% !TEX options=--shell-escape
\documentclass[10pt]{beamer}

\newtheorem{thm}{Theorem}
\usetheme[progressbar=frametitle]{metropolis}
\usepackage{appendixnumberbeamer}

\usepackage{booktabs}
\usepackage[scale=2]{ccicons}
\usepackage{minted}

\usepackage{pgfplots}
\usepgfplotslibrary{dateplot}

\usepackage{xspace}
\newcommand{\themename}{\textbf{\textsc{metropolis}}\xspace}

\title{CS1101S - Programming Methodology I}
\subtitle{Studio 4}
% \date{\today}
\date{}
\author{Theodore Leebrant}
\institute{Tutorial Group 8D}
% \titlegraphic{\hfill%\includegraphics[height=1.5cm]{logo.pdf}}

\begin{document}

\maketitle

\section[Admin Stuff]{Admininstrational Matters}

\begin{frame}[fragile]{Attendance Taking}
\begin{centering}
\textbf{Make sure you have taken your temperature.} \\
We will take photo when everyone is present. \\
\end{centering}
\end{frame}

\begin{frame}[fragile]{Mastery Checks}
\begin{centering}
Mastery Check 1 is open. \\
Please telegram me to set a schedule. \\
Details are in the slides for Studio 2. \\
\end{centering}
\end{frame}

\begin{frame}[fragile]{Events}
\begin{itemize}
  \item Contest \#1 - Closes 1st September
  \item Reading Assessment - 4 September, 1030 (1000) - 1114
\end{itemize}
\end{frame}

\begin{frame}[fragile]{About the Reading Assessment}
\begin{itemize}
  \item LumiNUS Quiz
  \item One sheet of A4 paper
  \item Prepare zoom, recording app, matric card, pencil and eraser / pen
  \item Practice questions available in LumiNUS
\end{itemize}
Question: do you want to go through the questions in past paper RA1? 
\end{frame}

\section[Recap]{Recap}

\begin{frame}[fragile]{Anonymous functions / Lambda expression}
  \begin{minted}{js}
const f = param => param + 1;

function f(param) {
  return param + 1;
}
f(1);
  \end{minted}
\end{frame}

\begin{frame}[fragile]{Function inside functions}
  \begin{minted}{js}
function factorial(n) {
  function fact_helper(n, res) {
    return n === 1
      ? res
      : fact_helper(n - 1, n * res);
  }

  return fact_helper(n, 1);
}
  \end{minted}
\end{frame}

\begin{frame}[fragile]{Higher order functions}
  Having functions as arguments and / or returned value. \\
  Non-programming example: integration / differentiation.
\end{frame}

\begin{frame}[fragile]{Higher order functions}
  Say I want the smaller of two things:
  \begin{minted}{js}
const min = (a, b) => a < b ? a : b
  \end{minted}
  But what if I want to compare two times? (e.g. in hh:mm format)
  \begin{minted}{js}
const min = (f, a, b) => f(a) < f(b) ? a : b
  \end{minted} 
\end{frame}

\begin{frame}[fragile]{Higher order functions}
  Partial application of function:
  \begin{minted}{js}
const sum = a => b => a + b;
const add_3 = sum(3);
add_3(100);
  \end{minted}
\end{frame}

\begin{frame}[fragile]{Scoping}
  \begin{itemize}
    \item We give names to things
    \item We may give many things the same name
    \item Which names refer to what? (we need context)
  \end{itemize}  
\end{frame}

\begin{frame}[fragile]{Scoping}
  \begin{itemize}
    \item A name occurence refers to the closest surrounding declaration
    \item Scope is the context where we find out names
    \item Most common context: blocks \verb|{ ... }|
    \item To find what a name refers to, look at the current scope, and then outwards. Take the first one you come across.
    \item Names in an outer scope can be hidden by definitions in an inner scope.
  \end{itemize}  
\end{frame}

\begin{frame}[fragile]{Scoping}
\href{https://share.sourceacademy.nus.edu.sg/motherfuckingscopes}{\underline{What makes a scope?}}
\end{frame}

\begin{frame}[fragile]{Scoping}
  \begin{minted}{js}
const hello = "world";
function n(hello) {
    const g = hello => display;
    g(hello)(hello);
    return g(hello);
}
n("hello")(hello);
  \end{minted}  
\end{frame}

\begin{frame}[fragile]{Scoping}
  \begin{minted}{js}
const n = 1;
{
    const n = 2;
    {
        const n = 3;
        {
            display(n);
        }
    }
}
  \end{minted}
\end{frame}

\begin{frame}[fragile]{Scoping}
\begin{minted}{js}
(f => x => f => x => f(x))
    (x => x + 1)(3)(x => x)(x => 2 * x + 3)
\end{minted}

\begin{minted}{js}
(a => a => a => a)(a => a)(a => a => a)(a => a => a)
\end{minted}

\begin{minted}{js}
(x => y => z => y(z)) (x => y => x(y)) (y => z => z) (1);
\end{minted}
\end{frame}

\begin{frame}[fragile]{Example of HOF: series function}
How to model a series $S(n) = \sum_{i=0}^n a_i x^n = a_0 x^0 + a_1 x^1 + \hdots$
\begin{minted}{js}
function series_generator(limit, coefficient) {
    // body here
}
function factorial(n) {
  return n * factorial(n - 1);
}
\end{minted}

Given that $e^x = \sum_{i=0}^{\infty} \frac{x^i}{i!}$

\begin{minted}{js}
const exp_coeff = n => 1 / factorial(n);
const exp_series = series_generator(10, exp_coefficient);
exp_series(2);
// Other way to call: series_generator(10, n => 1 / factorial(n))(2)
\end{minted}
\href{https://share.sourceacademy.nus.edu.sg/series}{\underline{Answer}} (contributed by Xiu Wen - T08D)
\end{frame}


\section[Studio Sheet (and Photo Taking)]{Studio Sheet}


\section[Additional Material]{Additional Material}

\begin{frame}[fragile]{Church Numerals - Functional Expressionism Quest}
\begin{itemize}
\item Peano arithmetic - how to describe the natural numbers
\item A set of axioms / postulates: (here, $\mathbb{N}$ denotes the set of natural numbers)
\begin{enumerate}
  \item 0 is a natural number
  \item $\forall x \in \mathbb{N}, x = x$ (equality is reflexive)
  \item $\forall x, y \in \mathbb{N}, x = y \rightarrow y = x$ (equality is symmetric)
  \item $\forall x, y, z \in \mathbb{N}, \Big(x=y \text{ and } y = z \Big) \Rightarrow x = z$ (equality is transitive)
  \item $\forall a, b; \Big(b \in \mathbb{N} \text{ and } a = b \Big) \Rightarrow a \in \mathbb{N}$ (closure under equality)
  \item $\forall n \in \mathbb{N}, S(n) \in \mathbb{N}$ where $S$ is the successor function (closure under $S$)
  \item $\forall m, n \in \mathbb{N}, m = n \iff S(m) = S(n)$ ($S$ is an injection)
  \item $\forall n \in \mathbb{N}, S(n) = 0$ is \textbf{false}. (No natural number whose successor is 0)
  \item If $K$ is a set such that $0 \in K$ and $\forall n \in \mathbb{N}, n \in K \Rightarrow S(n) \in K$ (induction)
\end{enumerate}
\end{itemize}
\end{frame}

\begin{frame}[fragile]{Church Numerals - Functional Expressionism Quest}
In lambda calculus, we can represent natural numbers too. \\
We define 0 as \verb|f => x => x| \\
1 is defined as \verb|f => x => f(x)| \\
2 is defined as \verb|f => x => f(f(x))| \\
... n is defined as \verb|f => x => f(f(f(...f(x)...)))| \\

The successor function is defined as applying \texttt{f} one more time. \\
In other words, \texttt{n => f => z => f(n(f)(z)); }
\end{frame}

\begin{frame}[fragile]{Church Numerals - Functional Expressionism Quest}
How do we decrement stuff? [Predecessor function] \\
Rules: pred(0) = 0 (can't go lower than that), otherwise pred(n) = m iff succ(m) = n. \\

To implement: [try google, I'm too tired to make these slides]
\end{frame}

\end{document}