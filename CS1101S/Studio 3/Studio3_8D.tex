% !TEX options=--shell-escape
\documentclass[10pt]{beamer}

\newtheorem{thm}{Theorem}
\usetheme[progressbar=frametitle]{metropolis}
\usepackage{appendixnumberbeamer}

\usepackage{booktabs}
\usepackage[scale=2]{ccicons}
\usepackage{minted}

\usepackage{pgfplots}
\usepgfplotslibrary{dateplot}

\usepackage{xspace}
\newcommand{\themename}{\textbf{\textsc{metropolis}}\xspace}

\title{CS1101S - Programming Methodology I}
\subtitle{Studio 3}
% \date{\today}
\date{}
\author{Theodore Leebrant}
\institute{Tutorial Group 8D}
% \titlegraphic{\hfill%\includegraphics[height=1.5cm]{logo.pdf}}

\begin{document}

\maketitle

\section[Admin Stuff]{Admininstrational Matters}

\begin{frame}[fragile]{Attendance Taking}
\begin{centering}
\textbf{Make sure you have taken your temperature.} \\
We will take photo when everyone is present. \\
\end{centering}
\end{frame}

\section[Regarding SourceAcademy tasks]{SourceAcademy Task Briefing}

\begin{frame}[fragile]{Paths}
  \begin{itemize}
    \item By now, you should have done 4 paths:
    \begin{enumerate}
      \item Elements of Programming (ungraded)
      \item Runes (140XP)
      \item Substitution model and recursion (140XP)
      \item Orders of Growth (160XP)
    \end{enumerate}
    \item These will be autograded at a later time.
    \item Paths are meant to assess your basic understanding of the topics taught in lectures and briefs. Please try to do them on time.
  \end{itemize}
\end{frame}

\begin{frame}[fragile]{Mission}
  \begin{itemize}
    \item Most, if not all of you, have completed the Rune Reading Mission.
    \item Some things to note:
    \begin{itemize}
      \item Read the comments 
      \item Please make your code readable! This is for your future projects as well as my sanity in marking.
      \item Read the Source styleguide. Some common mistakes include:
    \end{itemize}
  \end{itemize}
\end{frame}

\begin{frame}[fragile]{Common mistakes}
  \begin{itemize}
    \item Put your comments for a function outside the function, not inside the function body. \href{https://share.sourceacademy.nus.edu.sg/functioncomments}{\underline{Example}}
    \item For things other than functions, put the comments either on the line or after the line. \href{https://share.sourceacademy.nus.edu.sg/othercomments}{\underline{Example}}
    \item Indentation style: try to stick to one tab (not two) - this one is more forgivable, but you might get some trouble when it got to longer code, because
    \item You need to keep to 80 characters per line. \href{https://share.sourceacademy.nus.edu.sg/indentationonetab}{\underline{Example}}
    \item Also check the styleguide for indentation, as there are quite a bit on it. 
  \end{itemize}
\end{frame}

\section[Recap]{Recap}

\begin{frame}[fragile]{Substitution Model}
  \begin{itemize}
    \item Reason about programs
    \item Replace eligible sub-expression with result until you can't anymore (fully simplified/reduced) - some expressions are \textit{irreducible}.
    \item By performing repeated reductions, we can simplify and find the result of any given statement.
  \end{itemize}  
\end{frame}

\begin{frame}[fragile]{Applicative Order Reduction}
  Works like your normal arithmetic: evaluate from the left, and the deepest expression.
  \begin{itemize}
    \item Evaluate arguments first.
    \item Then substitute function call(s) with body.
  \end{itemize}
  Example:
  \begin{minted}{js}      
12345 % math_pow(10, math_floor(math_log10(12345)));
  \end{minted}   
  This will be evaluated as:
  \begin{minted}{js}
12345 % math_pow(10, math_floor(math_log10(12345)));
12345 % math_pow(10, math_floor(4.09...));
12345 % math_pow(10, 4);
12345 % 10000;
2345;
  \end{minted}
\end{frame}

\begin{frame}[fragile]{Normal Order Reduction}
  Do not evaluate arguments unless absolutely needed.
  \begin{itemize}
    \item Substitute function call(s) with body.
    \item Then evaluate the arguments.
  \end{itemize}
\end{frame}

\begin{frame}[fragile]{Normal Order Reduction (Example)}
Example:
  \begin{minted}{js}
  function sq(x) { 
      return x * x 
  };
  function dist(x, y) { 
      return math_sqrt(sq(x) + sq(y)) 
  };
  dist(1 + 5, 2 * 10);
  \end{minted}

  This program will be executed as follows:
  \begin{minted}{js}
dist(1 + 5, 2 * 10);
math_sqrt(sq(1 + 5) + sq(2 * 10))
math_sqrt((1 + 5) * (1 + 5) + (2 * 10) * (2 * 10))
math_sqrt((6) * (1 + 5) + (2 * 10) * (2 * 10))
math_sqrt((6) * (6) + (2 * 10) * (2 * 10))
...
  \end{minted}
\end{frame}

\begin{frame}[fragile]{Applicative vs Normal Order Reduction}
\begin{minted}{js}
function p() {
    return p();
}

function test(x, y) {
    return x === 0 ? 0 : y;
}

test(0, p());
\end{minted}
\end{frame}

\begin{frame}[fragile]{Recursive Problem-solving}
  \begin{enumerate}
    \item Base Case - the simplest problem you can think of
    \item Think of one slightly simpler function call. Can you build the desired function call from this one?
  \end{enumerate}
  Example: Tower of Hanoi

  The thing with recursion is belief! \\
  You need to assume that the smaller problem is correct. \\
  \textbf{Do not try to trace the recursion.}
\end{frame}

\begin{frame}[fragile]{Recursive vs Iterative Processes}
  \begin{itemize}
    \item Both involve calling the function inside the function body (i.e. recursion)
    \item A recursive process has \textbf{deferred operations}
    \item An iterative process does not
  \end{itemize}  
    Example: \href{https://share.sourceacademy.nus.edu.sg/itervsrec}{\underline{https://share.sourceacademy.nus.edu.sg/itervsrec}} \\
    (Use Stepper for visualisation)
\end{frame}

\begin{frame}[fragile]{Time Complexity}
Why do we care?
  \begin{itemize}
    \item Need an abstract way to talk about resources consumed
    \item The same for every languages, architecture, CPU
    \item Some abstract measure of time taken for the program to run.
    \item How do we characterize it?
    \begin{itemize}
      \item Number of operations performed.
      \item Number of “simple” operations performed for some input size.
      \item Simple operations:
      \begin{itemize}
        \item All arithmetic e.g. 4 * 5
        \item Memory read and write e.g. const a = 4;
        \item Conditionals e.g if (a === 4)
      \end{itemize}
    \end{itemize}
  \end{itemize}  
\end{frame}

\begin{frame}[fragile]{Space Complexity}
Why do we care?
  \begin{itemize}
    \item We don't want to run out of space :(
    \item Some abstract measure of space taken for the program to run is needed
    \item Characterized by maximum number of symbols created at one point in time (usually deferred operator)
  \end{itemize}
\end{frame}

\section[Studio Sheet (and Photo Taking)]{Studio Sheet}
\begin{frame}[fragile]{Question 1}
\begin{center}
Someone try to answer :D \\
\vspace{1em}
\includegraphics[width = 0.5\textwidth]{img/q2.png}
\end{center}
\end{frame}

\begin{frame}[fragile]{Question 1}
\begin{center}
Intuitive solution: \\
\vspace{1em}
\includegraphics[width = 0.5\textwidth]{img/q2.png}
\end{center}
\end{frame}

\begin{frame}[fragile]{Question 2}
\begin{center}
The proper solution: \href{https://share.sourceacademy.nus.edu.sg/q1proper}{\underline{here}} \\
\vspace{1em}
\includegraphics[width = 0.5\textwidth]{img/q1.png}
\end{center}
\end{frame}

\begin{frame}[fragile]{Question 2}
\begin{center}
The intuitive solution: (on the board) \\
\vspace{1em}
\includegraphics[width = 0.5\textwidth]{img/q1.png}
\end{center}
\end{frame}

\begin{frame}[fragile]{Question 3}
Take some time to think about the question, and someone try to explain.
\end{frame}

\begin{frame}[fragile]{Question 4}
\begin{verbatim}
    function power(b, n) {
        return n === 0 ? 1 : b * power(b, n - 1);
    }
\end{verbatim}
\begin{itemize}
    \item Iterative or recursive process? Can you write it in the other manner?
    \item Use the $\Theta$ notation to characterize the running time and space consumption of $power$ as the argument $n$ grows.
\end{itemize}
\end{frame}

\begin{frame}[fragile]{Question 4}
Consider the following example: \texttt{power(2, 3)}
\begin{minted}{js}
    power(2, 3)
    3 === 0 ? 1 : 2 * power(2, 2) // false
    2 * power(2, 2)
    2 * (2 === 0 ? 1 : 2 * power(2, 1)) // false
    2 * (2 * power(2, 1))
    2 * (2 * (1 === 0 ? 1 : 2 * power(2, 0))) // false
    2 * (2 * (2 * (power(2, 0))))
    2 * (2 * (2 * (0 === 0 ? 1 : power(2, -1)))) // true
    2 * (2 * (2 * (1)))
    2 * (2 * (2))
    2 * (4)
    8
\end{minted}
\end{frame}

\begin{frame}[fragile]{Question 5}
Given that 
\begin{equation*}
b^n = 
\begin{cases}
  b^{n/2}b^{n/2} &\text{if }n\text{ is even} \\
  b^{(n - 1)/2}b^{(n - 1)/2} &\text{if }n\text{ is even}
\end{cases}
\end{equation*}
\begin{itemize}
\item Implement a function $fast\_power(b, n)$ which computes $b^n$ in $O(log(n))$ time, where $n$ is a natural number. \href{https://share.sourceacademy.nus.edu.sg/fastpower}{[\underline{answer}]}\\

\item Can you extend this to integer powers? \\

\item Iterative or recursive process? Can you write it in the other manner?\\

\item Use the $\Theta$ notation to characterize the running time and space consumption of $fast\_power$ as the argument $n$ grows.
\end{itemize}

\end{frame}

\section[Additional Material]{Additional Material}

\begin{frame}{Big-O, Big-$\Omega$, Big-$\Theta$}
The intuitive definition of Big-O, Big-$\Omega$, Big-$\Theta$ are: \\
When the input is big enough,
\begin{table}
\centering
\begin{tabular}{|l|l|} 
\hline
$f(n) = O(g(n))$ & $f$ is bounded above by $g$  \\ 
\hline
$f(n) = \Omega(g(n))$       & $f $ is bounded below by $g$  \\ 
\hline
$f(n) = \Theta(g(n))$      & $f $ is bounded by $g$        \\
\hline
\end{tabular}
\end{table}
\centering
\end{frame}

\begin{frame}{Big-O, Big-$\Omega$, Big-$\Theta$}
\begin{centering}
\includegraphics[width = 0.5\textwidth]{img/bound.png} \\
\end{centering}
In this example, what is the relation between
\begin{itemize}
  \item $f$ and $g$
  \item $f$ and $h$
  \item $f$ and $k$
\end{itemize}
\end{frame}

\begin{frame}{Big-O, Big-$\Omega$, Big-$\Theta$}
\begin{figure}
\includegraphics[width = 0.5\textwidth]{img/complexity.png}
\caption{Graphs of commonly encountered time complexities}
\centering
\end{figure}
\end{frame}

\begin{frame}[fragile]{Big-O, Big-$\Omega$, Big-$\Theta$}
\textbf{Optional content - try if you like math, at this juncture.} \\
What are the complexities of:
\begin{itemize}
\item $4n^2 - n$
\item $5n^2 + n$
\item $\sqrt{n} + n$
\item $3^n n^2$
\end{itemize}

Verify the following:
\begin{itemize}
\item $log_5{n} = \Theta(\log{n})$
\item $10n \log{n} = O(n^2)$
\item $n^3 = O(2^n)$
\end{itemize}
\end{frame}

\begin{frame}[fragile]{Big-O, Big-$\Omega$, Big-$\Theta$}
\textbf{Optional content - try if you want some extra practice, at this juncture.} \\
What is the (space and time) complexity of:
\begin{minted}{js}
function factorial(n) {
    return n === 1
        ? 1
        : n * factorial(n - 1);
}
\end{minted}
\end{frame}

\begin{frame}[fragile]{Big-O, Big-$\Omega$, Big-$\Theta$}
\textbf{Optional content - try if you want some extra practice, at this juncture.} \\
What is the (space and time) complexity of:
\begin{minted}{js}
function helper(n, res) {
    return n === 1
        ? res
        : helper(n - 1, n * res);
}

function factorial(n) {
    return helper(n, 1);
}
\end{minted}
\end{frame}

\begin{frame}[fragile]{Big-O, Big-$\Omega$, Big-$\Theta$}
\textbf{Optional content - try if you want some extra practice, at this juncture.} \\
What is the (space and time) complexity of:
\begin{minted}{js}
function fizz(n) {
    if (n === 0) {
      return "done";
    } else {
        n % 3 === 0
            ? display("fizz")
            : n % 5 === 0
                ? display("buzz")
                : display(n);
        return fizz(n - 1);
    }
}
\end{minted}
\end{frame}

\begin{frame}[fragile]{Big-O, Big-$\Omega$, Big-$\Theta$}
\textbf{Optional content - try if you want some extra practice, at this juncture.} \\
Implement the following using both recursive and iterative processes:
\begin{itemize}
  \item Factorial
  \item Fibonacci
  \item Power
  \item GCD
  \item LCM
  \item Coin-change problem (covered this week in lecture!)
  \item ...(continued next slide)
\end{itemize}
\end{frame}


\begin{frame}[fragile]{Big-O, Big-$\Omega$, Big-$\Theta$}
\textbf{Optional content - try if you want some extra practice, at this juncture.} \\
Implement the following using both recursive and iterative processes:
\begin{itemize}
  \item Pascal triangle (covered this week in lecture!)
  \item Tower of Hanoi
  \item Permutations
  \item Combinations
  \item $sqrt(x)$ by Newton-Raphson method (or Newton's method)
  \item sine approximation (see \href{https://source-academy.github.io/sicp/chapters/1.2.3.html}{\underline{the book}})
\end{itemize}
Some cannot be implemented using iterative. \\
Analyze their space and time complexity. \\
Are there any closed form for these questions?
\end{frame}



\end{document}