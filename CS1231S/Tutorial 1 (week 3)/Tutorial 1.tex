\documentclass[10pt]{beamer}

\newtheorem{thm}{Theorem}
\usetheme[progressbar=frametitle]{metropolis}
\usepackage{appendixnumberbeamer}

\usepackage{booktabs}
\usepackage[scale=2]{ccicons}

\usepackage{pgfplots}
\usepgfplotslibrary{dateplot}

\usepackage{xspace}
\newcommand{\themename}{\textbf{\textsc{metropolis}}\xspace}

\title{CS1231S Discrete Structures}
\subtitle{Tutorial 1}
% \date{\today}
\date{}
\author{Theodore Leebrant}
\institute{Tutorial Group 3A}
% \titlegraphic{\hfill\includegraphics[height=1.5cm]{logo.pdf}}

\begin{document}

\maketitle

% \begin{frame}{Workflow}
%   \setbeamertemplate{section in toc}[sections numbered]
%   \tableofcontents%[hideallsubsections]
% \end{frame}

\section[Admin Stuff]{Admin}


\begin{frame}{Safety Measures}
\begin{itemize}
  \item Masks on at all time
  \item Stay 1.5m apart
  \item Passing objects require disinfection (e.g. markers)
  \item Take attendance and picture of seating arrangement
\end{itemize}
\end{frame}

\begin{frame}[fragile]{About Me}
  \begin{itemize}
    \item Theodore Leebrant
    \item Year 2 Computer Science \& Mathematics + USP
    \item Living in Cinnamon College
    \item Telegram: \href{https://t.me/kagamination}{@\underline{kagamination}}
    \item Email: \href{mailto:theodoreleebrant@u.nus.edu}{\underline{theodoreleebrant@u.nus.edu}}
  \end{itemize}
\end{frame}

\begin{frame}[fragile]{About This Tutoria}
\begin{itemize}
  \item Safe space
  \begin{itemize}
    \item Questions, mistakes, comments are welcome.
  \end{itemize}
  \item Expectations
  \begin{itemize}
    \item \textbf{Come to tutorial prepared for discussion!}
    \item (Try to) finish the tutorial questions.
    \item If still cannot, at least read the questions. 
  \end{itemize}
  \item Workflow
  \begin{itemize}
    \item We will go through the tutorial questions
    \item I'll stay back for any questions if needed, including discussion questions
  \end{itemize}
\end{itemize} 
\end{frame}


\begin{frame}[fragile]{Consultations}
If you need any consultation,
  \begin{itemize}
    \item PM me on Telegram (preferred) or drop an email
    \item Either group or 1-to-1 consultations are fine, keep it below 5 people.
    \item F2F (preferred) or online (through zoom/discord)
    \item Check for timing, at least 1 day ahead. Most free on Mondays.
  \end{itemize}
\end{frame}

\section[Recap]{Recap}

\begin{frame}[fragile]{Important Sets}
  \begin{itemize}
    \item $\mathbb{R}$: Set of all real numbers
    \item $\mathbb{Q}$: Set of all rational numbers (can be expressed as fractions)
    \item $\mathbb{Z}$: Set of all integers
    \item $\mathbb{N}$: Set of all natural numbers (non-negative integers)
  \end{itemize}
  Subscripts and superscripts used: ${\mathbb{R}}^+$, ${\mathbb{R}}^-$ ${\mathbb{R}}_{\leq 10}$
\end{frame}

\begin{frame}[fragile]{Important notations}
\begin{itemize}
\item $\in$: element of
\item $\forall$: for all
\item $\exists$: there exists
\item $\mid$: divisible by
\item $\nmid$: not divisible by
\end{itemize}
\end{frame}

\begin{frame}[fragile]{Writing Proofs}
For CS1231S:
\begin{itemize}
\item Introductory module
\item Clarity of presentation
\item One line for each step
\item One law per step
\end{itemize}
\end{frame}

\begin{frame}[fragile]{Important notations (Propositional and Conditional Logic)}
\begin{itemize}
\item $\lnot$, or $\sim$: not / negation
\item $\land$: and
\item $\lor$: or
\item $\rightarrow$: implies (if)
\item $\leftrightarrow$: if and only if (iff)
\end{itemize}
\end{frame}

\begin{frame}[fragile]{Order of operations}
\begin{enumerate}
\item $\lnot$ or $\sim$
\item $\land$ and $\lor$
\item $\rightarrow$
\end{enumerate}
Use brackets to eliminate ambiguity
\end{frame}

\begin{frame}[fragile]{Logical equivalences}
\begin{itemize}
  \item Commutativity: $p \lor q \equiv q \lor p$ (also works for $\land$)
  \item Associativity: $(p \lor q) \lor r \equiv p \lor (q \lor r)$ (also works for $\land$)
  \item Distributivity: $p \land (q \lor r) \equiv (p \land q) \lor (p \land r)$
  \item Identity law: $p \land \text{true} \equiv p$; $p \lor \text{false} \equiv p$
  \item Negation law: $p \lor \lnot p \equiv \text{true}$; $p \land \lnot p \equiv \text{false}$
  \item Idempotent law: $p \lor p \equiv p$; $p \land p \equiv p$
  \item Universal bound law: $p \lor \text{true} \equiv \text{true}$; $p \land \text{false} \equiv \text{false}$
  \item De Morgan's law: $\lnot (p \land q) \equiv \lnot p \lor \lnot q$
  \item Absorption law: $p \lor (p \land q) \equiv p$; $p \land (p \lor q) \equiv p$
  \item Negation of true and false
\end{itemize}
\end{frame}

\begin{frame}[fragile]{More logical equivalences}
\begin{itemize}
  \item Implication law: $p \rightarrow q \equiv \lnot p \lor q$
  \item Contrapositive: $p \rightarrow q \equiv \lnot q \rightarrow \lnot p$
  \item Biconditional equivalence: $p \leftrightarrow q \equiv (p \rightarrow q) \land (q \rightarrow p)$
\end{itemize}
\end{frame}

\begin{frame}[fragile]{Converse and Inverse}
Consider $p \rightarrow q$.
\begin{itemize}
  \item Converse: $q \rightarrow p$
  \item Inverse: $\lnot p \rightarrow \lnot q$
\end{itemize}
\end{frame}

\begin{frame}[fragile]{Rules of Inference}
\begin{itemize}
  \item Modus ponens
  \item Modus tollense
  \item Generalisation
  \item Specialisation
  \item Conjunction
  \item Elimination
  \item Transitivity
  \item Proof by Division into cases
  \item Contradiction Rule 
\end{itemize}
\end{frame}

\begin{frame}[fragile]{Rule of thumb on how to prove a logical statement}
\begin{itemize}
\item Put the entire statement into consideration (e.g. Tutorial qn D2(a): do not just consider the left of the implication)
\item You can try to sketch something, but answer needs to be step-by-step
\item If all else fails, truth table (not recommended)
\end{itemize}
\end{frame}

\section[Tutorial Questions (and Photo Taking)]{Tutorial Questions}

\begin{frame}[fragile]{Question 1a}
    Claim: Assuming $a \in \mathbb{R}$, the negation of $(1 < a < 5)$ is $(1 \geq a \geq 5)$.
    \begin{center}
        $\sim (1 < a < 5)$ \\
        $\sim ((1 < a) \land (a < 5))$ \\
        $(\sim (1 < a)) \lor (\sim (a < 5))$ \\
        $(1 \geq a) \lor (a \geq 5)$ \\
    \end{center}
    But $(1 \geq  a \geq 5)$ means $(1 \geq a) \land (a \geq 5) \not\equiv (1 \geq a) \lor (a \geq 5)$. \\
    Hence, (a) is \textbf{false}.
\end{frame}

\begin{frame}[fragile]{Question 1b}
    Claim: The two statements are logically equivalent:
    \begin{enumerate}
        \item “he’s welcome to come along only if he behaves himself”
        \item “if he behaves himself then he’s welcome to come along”
    \end{enumerate}
    Let $p$ be "he's welcome to come along" and $q$ be "he behaves himself". \\
    Recall:
    \begin{itemize}
        \item $p$ only if $q$ means $p \rightarrow q$
        \item if $p$ then $q$ means $p \rightarrow q$
    \end{itemize}
    So Statement 1 $\equiv p \rightarrow q$ \\
    but Statement 2 $\equiv q \rightarrow p$. \\
    Hence, (b) is \textbf{false}.
\end{frame}

\begin{frame}[fragile]{Question 2}
a) $$\sim a \land ( \sim a \rightarrow (a \land b))$$
    \begin{align*}
        &\equiv \ \sim a \land ( \sim (\sim a) \lor (a \land b)) &\text{ by implication law} \\
        &\equiv \ \sim a \land ( a \lor (a \land b)) &\text{ by double negative law} \\
        &\equiv \ \sim a \land a &\text{ by absorption law} \\
        &\equiv a \ \land \sim a &\text{ by commutative law} \\
        &\equiv \text{\textbf{false}} &\text{ by negation law} 
    \end{align*}
\end{frame}

\begin{frame}[fragile]{Question 2}
b)
\[p \ \lor \sim q \rightarrow q \]
    \begin{align*}
        &\equiv \ \sim (p \ \lor \sim q) \lor q &\text{ by implication law} \\
        &\equiv \ (\sim p \ \land \sim (\sim q)) \lor q &\text{ by De Morgan's law} \\
        &\equiv \ (\sim p \land q) \lor q &\text{ by double negative law} \\
        &\equiv \ q \lor (\sim p \land q) &\text{ by commutative law} \\
        &\equiv \ q \lor (q \ \land \sim p) &\text{ by commutative law} \\
        &\equiv \ q &\text{ by absorption law}
    \end{align*}
\end{frame}

\begin{frame}[fragile]{Question 2}
c)
\[\sim (p \ \lor \sim q) \lor (\sim p \ \land \sim q)\]
    \begin{align*}
     &\equiv(\sim p \lor \sim (\sim q)) \lor (\sim p \land \sim q) &\text{ by De Morgan's Law} \\
     &\equiv(\sim p \lor q) \lor (\sim p \land \sim q) &\text{ by double negative law} \\
     &\equiv\sim p \lor (q \land \sim q) &\text{ by distributive law} \\
     &\equiv\sim p \lor \text{\textbf{true}} &\text{ by distributive law} \\
     &\equiv\sim p &\text{ by identity law}
    \end{align*}
\end{frame}

\begin{frame}[fragile]{Question 2}
d)
\[(p \rightarrow q) \rightarrow r\]
    \begin{align}
        &\equiv (\sim p \lor q) \rightarrow r &\text{ by implication law} \\
        &\equiv (\sim(\sim p \lor q)) \lor r &\text{ by implication law} \\
        &\equiv (\sim(\sim p) \,\land \sim q) \lor r &\text{ by De Morgan's law} \\
        &\equiv (p \,\land \sim q) \lor r &\text{ by the double negative law}
    \end{align}
\end{frame}

\begin{frame}[fragile]{Question 3}
Answer: $(p \rightarrow q) \rightarrow r$ and $p \rightarrow (q \rightarrow r)$ are \textbf{not} logically equivalent.\\
To show this, we just need to get one counterexample from the following truth table:
\begin{table} 
\centering 
\begin{tabular}{|c|c|c|c|c|c|c|} 

\hline $p$ & $q$ & $r$ & $p \rightarrow q$ & $(p \rightarrow q) \rightarrow r$ & $q \rightarrow r$ & $p \rightarrow (q \rightarrow r)$ \\

 
\hline \textcolor{blue}{true} & 
\textcolor{blue}{true} & 
\textcolor{blue}{true} & 
\textcolor{blue}{true} & 
\textcolor{blue}{true} & 
\textcolor{blue}{true} & 
\textcolor{blue}{true} \\

 
\hline \textcolor{blue}{true} & 
\textcolor{blue}{true} & 
\textcolor{red}{false} & 
\textcolor{blue}{true} & 
\textcolor{red}{false} & 
\textcolor{red}{false} & 
\textcolor{red}{false} \\

 
\hline \textcolor{blue}{true} & 
\textcolor{red}{false} & 
\textcolor{blue}{true} & 
\textcolor{red}{false} & 
\textcolor{blue}{true} & 
\textcolor{blue}{true} & 
\textcolor{blue}{true} \\

 
\hline \textcolor{blue}{true} & 
\textcolor{red}{false} & 
\textcolor{red}{false} & 
\textcolor{red}{false} & 
\textcolor{blue}{true} & 
\textcolor{blue}{true} & 
\textcolor{blue}{true} \\

 
\hline 
\textcolor{red}{false} & 
\textcolor{blue}{true} & 
\textcolor{blue}{true} & 
\textcolor{blue}{true} & 
\textcolor{blue}{true} & 
\textcolor{blue}{true} & 
\textcolor{blue}{true} \\

 
\hline 
\textcolor{red}{false} & 
\textcolor{blue}{true} & 
\textcolor{red}{false} & 
\textcolor{blue}{true} & 
\textcolor{red}{\underline{\textbf{false}}} & 
\textcolor{red}{false} & 
\textcolor{blue}{\underline{\textbf{true}}} \\

 
\hline 
\textcolor{red}{false} & 
\textcolor{red}{false} & 
\textcolor{blue}{true} & 
\textcolor{blue}{true} & 
\textcolor{blue}{true} & 
\textcolor{blue}{true} & 
\textcolor{blue}{true} \\

 
\hline 
\textcolor{red}{false} & 
\textcolor{red}{false} & 
\textcolor{red}{false} & 
\textcolor{blue}{true} & 
\textcolor{red}{\underline{\textbf{false}}} & 
\textcolor{blue}{true} & 
\textcolor{blue}{\underline{\textbf{true}}} \\


\hline 
\end{tabular} 
\end{table}
\end{frame}

\begin{frame}[fragile]{Question 4}
\begin{center}
    "The rule says that to qualify for the draw, SAFRA-DBS credit card holders must
'charge a minimum of S\$50 nett to their card during the Qualifying Period', which is
1 July to 30 September 2017."
\end{center}
    Let $C =$ "Charge a minimum of S\$50 nett" \\
    $P =$ "Charge during the Qualifying Period" \\
    $W =$ "Win 100,000 AirAsia Miles"
\end{frame}

\begin{frame}[fragile]{Question 4a}
    Write a conditional statement using C, P and W that describes the rule above. \\
    Note that the qualifying conditions are \textbf{necessary but not sufficient} conditions. Thus, $C$ and $P$ are necessary for $W$, which translates to:
    \begin{center}
        if $W$ then $(C \land P)$ \\
        $W \rightarrow (C \land P)$
    \end{center}
\end{frame}

\begin{frame}[fragile]{Question 4b}
    Write the converse, inverse, contrapositive and negation forms of the statement in part (a).
    \begin{align*}
        &\text{Statement: } &W \rightarrow (C \land P) \\
        &\text{Converse: } &(C \land P) \rightarrow W \\
        &\text{Inverse: } &\sim W \rightarrow \,\sim(C \land P) \\
        &\text{Contrapositive: } &\sim(C \land P) \rightarrow \,\sim W \\
        &\text{Negation: } &\sim(W \rightarrow (C \land P))
    \end{align*}
\end{frame}

\begin{frame}[fragile]{Question 5}
Idea: for a valid conditional statement, the transitive rule of inference holds. \\
If $(p \rightarrow q)$ and $(q \rightarrow r)$, then $(q \rightarrow r)$. \\
i.e. $((p \rightarrow q) \land (q \rightarrow r)) \rightarrow (p \rightarrow r)$ would be always true (tautology)

We have three 'alternative' definition of the conditional statement. To show that they are not correct, we can show that the transitive rule of inference does not hold: in this case, showing a counterexample is sufficient.
\end{frame}



\begin{frame}[fragile]{Question 5 (Alternative conditional statement a)}
Using the first alternative definition of conditional statement,
\begin{table}
\centering
\begin{tabular}{|c|c|c|c|c|c|} 
\hline
$p$                     & $q$                     & $r$                     & $p \rightarrow q$       & $q \rightarrow r$       & $p \rightarrow r$        \\ 
\hline
\textcolor{blue}{true}  & \textcolor{blue}{true}  & \textcolor{blue}{true}  & \textcolor{blue}{true}  & \textcolor{blue}{true}  & \textcolor{blue}{true}   \\ 
\hline
\textcolor{blue}{true}  & \textcolor{blue}{true}  & \textcolor{red}{false}  & \textcolor{blue}{true}  & \textcolor{red}{false}  & \textcolor{red}{false}   \\ 
\hline
\textcolor{blue}{true}  & \textcolor{red}{false}  & \textcolor{blue}{true}  & \textcolor{red}{false}  & \textcolor{red}{false}  & \textcolor{blue}{true}   \\ 
\hline
\textcolor{blue}{true}  & \textcolor{red}{false}  & \textcolor{red}{false}  & \textcolor{red}{false}  & \textcolor{red}{false}  & \textcolor{red}{false}   \\ 
\hline
\textcolor{red}{false}  & \textcolor{blue}{true}  & \textcolor{blue}{true}  & \textcolor{red}{false}  & \textcolor{blue}{true}  & \textcolor{red}{false}   \\ 
\hline
\textcolor{red}{false}  & \textcolor{blue}{true}  & \textcolor{red}{false}  & \textcolor{red}{false}  & \textcolor{red}{false}  & \textcolor{red}{false}   \\ 
\hline
\textcolor{red}{false}  & \textcolor{red}{false}  & \textcolor{blue}{true}  & \textcolor{red}{false}  & \textcolor{red}{false}  & \textcolor{red}{false}   \\ 
\hline
\textcolor{red}{false}  & \textcolor{red}{false}  & \textcolor{red}{false}  & \textcolor{red}{false}  & \textcolor{red}{false}  & \textcolor{red}{false}   \\
\hline
\end{tabular}
\end{table}
\end{frame}

\begin{frame}[fragile]{Question 5 (Alternative conditional statement a)}
Therefore,
\begin{table}
\centering
\begin{tabular}{|c|c|} 
\hline
$(p \rightarrow q) \land (q \rightarrow r)$  & $((p \rightarrow q) \land (q \rightarrow r)) \rightarrow (p \rightarrow r) $   \\ 
\hline
\textcolor{blue}{true}                       & \textcolor{blue}{true}                                                         \\ 
\hline
\textcolor{red}{false}                       & \textbf{\textcolor{red}{false}}                                                \\ 
\hline
\textcolor{red}{false}                       & \textbf{\textcolor{red}{false}}                                                \\ 
\hline
\textcolor{red}{false}                       & \textbf{\textcolor{red}{false}}                                                \\ 
\hline
\textcolor{red}{false}                       & \textbf{\textcolor{red}{false}}                                                \\ 
\hline
\textcolor{red}{false}                       & \textbf{\textcolor{red}{false}}                                                \\ 
\hline
\textcolor{red}{false}                       & \textbf{\textcolor{red}{false}}                                                \\ 
\hline
\textcolor{red}{false}                       & \textbf{\textcolor{red}{false}}                                                \\
\hline
\end{tabular}
\end{table}
\end{frame}

\begin{frame}[fragile]{Question 5 (alternative conditional statement b)}
Using the second alternative definition of conditional statement,
\begin{table}
\centering
\begin{tabular}{|c|c|c|c|c|c|} 
\hline
 $p$                    & $q$                     & $r$                     & $p \rightarrow q$       & $q \rightarrow r$       & $p \rightarrow r$        \\ 
\hline
\textcolor{blue}{true}  & \textcolor{blue}{true}  & \textcolor{blue}{true}  & \textcolor{blue}{true}  & \textcolor{blue}{true}  & \textcolor{blue}{true}   \\ 
\hline
\textcolor{blue}{true}  & \textcolor{blue}{true}  & \textcolor{red}{false}  & \textcolor{blue}{true}  & \textcolor{red}{false}  & \textcolor{red}{false}   \\ 
\hline
\textcolor{blue}{true}  & \textcolor{red}{false}  & \textcolor{blue}{true}  & \textcolor{red}{false}  & \textcolor{blue}{true}  & \textcolor{blue}{true}   \\ 
\hline
\textcolor{blue}{true}  & \textcolor{red}{false}  & \textcolor{red}{false}  & \textcolor{red}{false}  & \textcolor{red}{false}  & \textcolor{red}{false}   \\ 
\hline
\textcolor{red}{false}  & \textcolor{blue}{true}  & \textcolor{blue}{true}  & \textcolor{blue}{true}  & \textcolor{blue}{true}  & \textcolor{blue}{true}   \\ 
\hline
\textcolor{red}{false}  & \textcolor{blue}{true}  & \textcolor{red}{false}  & \textcolor{blue}{true}  & \textcolor{red}{false}  & \textcolor{red}{false}   \\ 
\hline
\textcolor{red}{false}  & \textcolor{red}{false}  & \textcolor{blue}{true}  & \textcolor{red}{false}  & \textcolor{blue}{true}  & \textcolor{blue}{true}   \\ 
\hline
\textcolor{red}{false}  & \textcolor{red}{false}  & \textcolor{red}{false}  & \textcolor{red}{false}  & \textcolor{red}{false}  & \textcolor{red}{false}   \\
\hline
\end{tabular}
\end{table}
\end{frame}

\begin{frame}[fragile]{Question 5 (Alternative conditional statement b)}
Therefore,
\begin{table}
\centering
\begin{tabular}{|c|c|} 
\hline
$(p \rightarrow q) \land (q \rightarrow r)$  & $((p \rightarrow q) \land (q \rightarrow r)) \rightarrow (p \rightarrow r) $   \\ 
\hline
\textcolor{blue}{true}                       & \textcolor{blue}{true}                                                         \\ 
\hline
\textcolor{red}{false}                       & \textcolor{red}{\textbf{false}}                                                         \\ 
\hline
\textcolor{red}{false}                       & \textcolor{blue}{true}                                                         \\ 
\hline
\textcolor{red}{false}                       & \textcolor{red}{\textbf{false}}                                                         \\ 
\hline
\textcolor{blue}{true}                       & \textcolor{blue}{true}                                                         \\ 
\hline
\textcolor{red}{false}                       & \textcolor{red}{\textbf{false}}                                                         \\ 
\hline
\textcolor{red}{false}                       & \textcolor{blue}{true}                                                         \\ 
\hline
\textcolor{red}{false}                       & \textcolor{red}{\textbf{false}}                                                         \\
\hline
\end{tabular}
\end{table}
\end{frame}

\begin{frame}[fragile]{Question 5 (Alternative conditional statement c)}
Using the third alternative definition of conditional statement,
\begin{table}
\centering
\begin{tabular}{|c|c|c|c|c|} 
\hline
 $p$                    & $q$                     & $r$                     & $p \rightarrow q$       & $q \rightarrow r$        \\ 
\hline
\textcolor{blue}{true}  & \textcolor{blue}{true}  & \textcolor{blue}{true}  & \textcolor{blue}{true}  & \textcolor{blue}{true}   \\ 
\hline
\textcolor{blue}{true}  & \textcolor{blue}{true}  & \textcolor{red}{false}  & \textcolor{blue}{true}  & \textcolor{red}{false}   \\ 
\hline
\textcolor{blue}{true}  & \textcolor{red}{false}  & \textcolor{blue}{true}  & \textcolor{red}{false}  & \textcolor{red}{false}   \\ 
\hline
\textcolor{blue}{true}  & \textcolor{red}{false}  & \textcolor{red}{false}  & \textcolor{red}{false}  & \textcolor{blue}{true}   \\ 
\hline
\textcolor{red}{false}  & \textcolor{blue}{true}  & \textcolor{blue}{true}  & \textcolor{red}{false}  & \textcolor{blue}{true}   \\ 
\hline
\textcolor{red}{false}  & \textcolor{blue}{true}  & \textcolor{red}{false}  & \textcolor{red}{false}  & \textcolor{red}{false}   \\ 
\hline
\textcolor{red}{false}  & \textcolor{red}{false}  & \textcolor{blue}{true}  & \textcolor{blue}{true}  & \textcolor{red}{false}   \\ 
\hline
\textcolor{red}{false}  & \textcolor{red}{false}  & \textcolor{red}{false}  & \textcolor{blue}{true}  & \textcolor{blue}{true}   \\
\hline
\end{tabular}
\end{table}
\end{frame}

\begin{frame}[fragile]{Question 5 (Alternative conditional statement c)}
Therefore,
\begin{table}
\centering
\begin{tabular}{|c|c|} 
\hline
$(p \rightarrow q) \land (q \rightarrow r)$  & $((p \rightarrow q) \land (q \rightarrow r)) \rightarrow (p \rightarrow r) $   \\ 
\hline
\textcolor{blue}{true}                       & \textcolor{blue}{true}                                                         \\ 
\hline
\textcolor{red}{false}                       & \textcolor{blue}{true}                                                         \\ 
\hline
\textcolor{red}{false}                       & \textcolor{blue}{true}                                                         \\ 
\hline
\textcolor{red}{false}                       & \textcolor{red}{\textbf{false}}                                                         \\ 
\hline
\textcolor{red}{false}                       & \textcolor{red}{\textbf{false}}                                                         \\ 
\hline
\textcolor{red}{false}                       & \textcolor{blue}{true}                                                         \\ 
\hline
\textcolor{red}{false}                       & \textcolor{blue}{true}                                                         \\ 
\hline
\textcolor{blue}{true}                       & \textcolor{blue}{true}                                                         \\
\hline
\end{tabular}
\end{table}
\end{frame}


\begin{frame}[fragile]{Question 6a}
    Sandra knows Java and Sandra knows C++. \\
    $\therefore$ Sandra knows C++. \par
    Let $p =$ "Sandra knows Java" \\
    Let $q =$ "Sandra knows C++" 
\begin{align*}
    &p \land q &\text{(premise)} \\
    &\therefore q &\text{(valid by specialisation)}
\end{align*}
\end{frame}

\begin{frame}[fragile]{Question 6b}
    If at least one of these two numbers is divisible by 6, then the product of these two numbers is divisible by 6. \\
    Neither of these two numbers is divisible by 6. \\
    $\therefore$ The product of these two numbers is not divisible by 6. \par
    Let $p =$ "the first number is divisible by 6" \\
    Let $q =$ "the second number is divisible by 6" \\
    Let $r =$ "the product of these two numbers is divisible by 6"
\begin{align*}
    &p \lor q \rightarrow r &\text{(premise)} \\
    &\sim p \,\land \sim q &\text{(premise)} \\
    &\therefore \,\sim r &\text{(invalid: inverse error)}
\end{align*}
\end{frame}

\begin{frame}[fragile]{Question 6c}
    If there are as many rational numbers as there are irrational numbers, then the set of all irrational numbers is infinite \\
    The set of all irrational numbers is infinite. \\
    $\therefore$ There are as many rational numbers as there are irrational numbers. \par
    Let $p =$ "there are as many rational numbers as there are irrational numbers" \\
    Let $q =$ "the set of all irrational numbers is infinite" 
\begin{align*}
    &p \rightarrow q &\text{(premise)} \\
    &q &\text{(premise)} \\
    &\therefore p &\text{(invalid: converse error)}
\end{align*}
\end{frame}

\begin{frame}[fragile]{Question 6d}
    If I get a Christmas bonus, I’ll buy a stereo. \\
    If I sell my motorcycle, I’ll buy a stereo. \\
    $\therefore$ If I get a Christmas bonus or I sell my motorcycle, I’ll buy a stereo. \par
    Let $p =$ "I get a Christmas bonus" \\
    Let $q =$ "I sell my motorcycle" \\
    Let $r =$ "I’ll buy a stereo"
\begin{align*}
    &p \rightarrow r &\text{(premise)} \\
    &q \rightarrow r &\text{(premise)} \\
    &(p \rightarrow r) \land (q \rightarrow r) &\text{(by conjunction)} \\
    &(\sim p \lor r) \land (\sim q \lor r) &\text{(by implication law)} \\
    &(\sim p \,\land \sim q) \lor r &\text{(by distributive law)} \\
    & (p \lor q) \rightarrow r &\text{(by implication law)}
\end{align*}
\end{frame}

\begin{frame}[fragile]{Question 7}
    Prove that $\exists x,y,z \in \mathbb{Z}_{>10}$ such that $x^2 + y^2 = z^2$. What is your proof called? What are these values called? \par
    \textit{Proof}
\begin{enumerate}
    \item Let $x = 11, y = 60, z = 61$.
    \item Then $x, y, z \in \mathbb{Z}_{>10}$ and $x^2 + y^2 = 11^2 + 60^2 = 121 + 3600 = 3721 = 61^2$.
    \item Thus $\exists x,y,z \in \mathbb{Z}_{>10}$ such that $x^2 + y^2 = z^2$. $\qed$
\end{enumerate}
    This is proof by construction. The values are called Pythagorean triples.
\end{frame}

\begin{frame}[fragile]{Question 8}
    The island of Wantuutrewan is inhabited by two types of people: knights who always tell the truth and knaves who always lie. You visit the island and have the following encounters with the natives. \par
    Two natives $C$ and $D$ speak to you:
    \begin{itemize}
        \item $C$ says: $D$ is a knave.
        \item $D$ says: $C$ is a knave.
    \end{itemize}
    What are $C$ and $D$?
\end{frame}

\begin{frame}[fragile]{Question 8 (cont.)}
    \textit{Proof} (by exhaustion)
\begin{enumerate}
    \item If $C$ is a knight:
    \begin{enumerate}
        \item What $C$ says is true. (by definition of knight)
        \item $\therefore D$ is a knave. (what $C$ says)
    \end{enumerate}
    \item If $C$ is not a knight:
    \begin{enumerate}
        \item Then $C$ is a knave. (one is either a knight or a knave)
        \item $\therefore$ what $C$ says is false. (by definition of knave)
        \item $\therefore D$ is not a knave. (negation of what $C$ says)
        \item $\therefore D$ is a knight (one is either a knight or a knave)
    \end{enumerate}
    \item In both cases, there is one knight and one knave. $\qed$
\end{enumerate}
\end{frame}

\begin{frame}[fragile]{Question 9}
    Prove the following statement:
\begin{center}
    The product of any two odd integers is an odd integer.
\end{center}
    \textit{Proof} (direct)
\begin{enumerate}
    \item Take any two odd integers $m, n$.
    \item Then $m = 2k + 1$ and $n = 2p + 1$ for $k, p \in \mathbb{Z}$\\
    (by definition of odd numbers)
    \item $mn = (2k + 1)(2p + 1) = 4kp + 2k + 2p + 1 = 2(2kp + k + p) + 1$\\
    (by basic algebra)
    \item Let $q = 2kp + k + p$. Then $q \in \mathbb{Z}$.\\
    (by closure of integers under $+$ and $\times$)
    \item Then $mn = 2q + 1$. So $mn$ is odd.\\
    (by definition of odd numbers)
    \item Therefore, the product of any two odd integers is an odd integer. $\qed$
\end{enumerate}
\end{frame}

\begin{frame}[fragile]{Question 10a}
    The problem with Smart's proof: \par
    In Line 5, Smart claims that $\sqrt{4k^2 + 4k + 4m^2 + 2}$ not an integer, but did not provide a proof nor cite any theorem to support his claim. Thus, Smart’s proof is incomplete, which means we cannot be sure if it is correct.
\end{frame}

\begin{frame}[fragile]{Question 10b}
    \textit{Proof} (by contraposition)
\begin{enumerate}
    \item Take the contraposition: if $a, b$ are both odd, then $a^2 + b^2 \neq c^2$.
    \item Suppose $a, b$ are both odd.
    \item Then $\exists k, m \in \mathbb{Z} \ni a = 2k + 1$ and $b = 2m + 1$. (by definition of odd numbers)
    \item $a^2 + b^2 = (2k+ 1)^2 + (2m + 1)^2 = 4k^2 + 4k + 4m^2 + 4m + 2$ (by basic algebra)
    \item Let $z = k^2 + k + m^2 + m$. Then $a^2 + b^2 = 4z + 2 = 2(2z + 1)$. (by basic algebra)
    \item Note that $z$ and $(2z + 1)$ are both integers. (by closure of integers under addition and multiplication)
    \item Hence $a^2 + b^2$ is even. (by definition of even numbers)
    \item Moreover, since $a^2 + b^2 = 4z + 2$, it follows that $a^2 + b^2$ has remainder 2 when divided by 4.
\end{enumerate}
\end{frame}

\begin{frame}[fragile]{Question 10b (cont.)}
\begin{enumerate}
    \item[9.] Now, $c$ is either odd or even.
    \begin{enumerate}
        \item[9.1] Case 1: $c$ is odd.
        \begin{enumerate}
            \item[9.1.1] Then $c^2$ is odd. (by Question 9)
            \item[9.1.2] Then $c^2 \neq a^2 + b^2$ since RHS is even (from line 7).
        \end{enumerate}
        \item[9.2] Case 2: $c$ is even.
        \begin{enumerate}
            \item[9.2.1] Then $\exists p \in \mathbb{Z} \ni c = 2p$ (by definition of even numbers)
            \item[9.2.2] Then $c^2 = 4p^2$. (by basic algebra)
            \item[9.2.3] Hence $c^2$ has a remainder of 0 when divided by 4.
            \item[9.2.4] Therefore $c^2 \neq a^2 + b^2$ since RHS has remainder 2 when divided by 4. (from line 8)
        \end{enumerate}
    \end{enumerate}
    \item[10.] In all cases, $c^2 \neq a^2 + b^2$.
    \item[11.] Therefore, by contraposition, the original statement is true. $\qed$
\end{enumerate}
\end{frame}

\section[Feedback / Questions]{Feedback / Questions}

\begin{frame}[fragile]{Feedback}
\centering
\includegraphics[width=0.7\textwidth]{frame(1).png}
\end{frame}



    

\end{document}
